\documentclass[11pt,preprint, authoryear]{elsarticle}

\usepackage{lmodern}
%%%% My spacing
\usepackage{setspace}
\setstretch{1.2}
\DeclareMathSizes{12}{14}{10}{10}

% Wrap around which gives all figures included the [H] command, or places it "here". This can be tedious to code in Rmarkdown.
\usepackage{float}
\let\origfigure\figure
\let\endorigfigure\endfigure
\renewenvironment{figure}[1][2] {
    \expandafter\origfigure\expandafter[H]
} {
    \endorigfigure
}

\let\origtable\table
\let\endorigtable\endtable
\renewenvironment{table}[1][2] {
    \expandafter\origtable\expandafter[H]
} {
    \endorigtable
}


\usepackage{ifxetex,ifluatex}
\usepackage{fixltx2e} % provides \textsubscript
\ifnum 0\ifxetex 1\fi\ifluatex 1\fi=0 % if pdftex
  \usepackage[T1]{fontenc}
  \usepackage[utf8]{inputenc}
\else % if luatex or xelatex
  \ifxetex
    \usepackage{mathspec}
    \usepackage{xltxtra,xunicode}
  \else
    \usepackage{fontspec}
  \fi
  \defaultfontfeatures{Mapping=tex-text,Scale=MatchLowercase}
  \newcommand{\euro}{€}
\fi

\usepackage{amssymb, amsmath, amsthm, amsfonts}

\def\bibsection{\section*{References}} %%% Make "References" appear before bibliography


\usepackage[round]{natbib}
\bibliographystyle{plainnat}

\usepackage{longtable}
\usepackage[margin=2.3cm,bottom=2cm,top=2.5cm, includefoot]{geometry}
\usepackage{fancyhdr}
\usepackage[bottom, hang, flushmargin]{footmisc}
\usepackage{graphicx}
\numberwithin{equation}{section}
\numberwithin{figure}{section}
\numberwithin{table}{section}
\setlength{\parindent}{0cm}
\setlength{\parskip}{1.3ex plus 0.5ex minus 0.3ex}
\usepackage{textcomp}
\renewcommand{\headrulewidth}{0.2pt}
\renewcommand{\footrulewidth}{0.3pt}

\usepackage{array}
\newcolumntype{x}[1]{>{\centering\arraybackslash\hspace{0pt}}p{#1}}

%%%%  Remove the "preprint submitted to" part. Don't worry about this either, it just looks better without it:
\makeatletter
\def\ps@pprintTitle{%
  \let\@oddhead\@empty
  \let\@evenhead\@empty
  \let\@oddfoot\@empty
  \let\@evenfoot\@oddfoot
}
\makeatother

 \def\tightlist{} % This allows for subbullets!

\usepackage{hyperref}
\hypersetup{breaklinks=true,
            bookmarks=true,
            colorlinks=true,
            citecolor=blue,
            urlcolor=blue,
            linkcolor=blue,
            pdfborder={0 0 0}}


% The following packages allow huxtable to work:
\usepackage{siunitx}
\usepackage{multirow}
\usepackage{hhline}
\usepackage{calc}
\usepackage{tabularx}
\usepackage{booktabs}
\usepackage{caption}


\urlstyle{same}  % don't use monospace font for urls
\setlength{\parindent}{0pt}
\setlength{\parskip}{6pt plus 2pt minus 1pt}
\setlength{\emergencystretch}{3em}  % prevent overfull lines
\setcounter{secnumdepth}{5}

%%% Use protect on footnotes to avoid problems with footnotes in titles
\let\rmarkdownfootnote\footnote%
\def\footnote{\protect\rmarkdownfootnote}
\IfFileExists{upquote.sty}{\usepackage{upquote}}{}

%%% Include extra packages specified by user

%%% Hard setting column skips for reports - this ensures greater consistency and control over the length settings in the document.
%% page layout
%% paragraphs
\setlength{\baselineskip}{12pt plus 0pt minus 0pt}
\setlength{\parskip}{12pt plus 0pt minus 0pt}
\setlength{\parindent}{0pt plus 0pt minus 0pt}
%% floats
\setlength{\floatsep}{12pt plus 0 pt minus 0pt}
\setlength{\textfloatsep}{20pt plus 0pt minus 0pt}
\setlength{\intextsep}{14pt plus 0pt minus 0pt}
\setlength{\dbltextfloatsep}{20pt plus 0pt minus 0pt}
\setlength{\dblfloatsep}{14pt plus 0pt minus 0pt}
%% maths
\setlength{\abovedisplayskip}{12pt plus 0pt minus 0pt}
\setlength{\belowdisplayskip}{12pt plus 0pt minus 0pt}
%% lists
\setlength{\topsep}{10pt plus 0pt minus 0pt}
\setlength{\partopsep}{3pt plus 0pt minus 0pt}
\setlength{\itemsep}{5pt plus 0pt minus 0pt}
\setlength{\labelsep}{8mm plus 0mm minus 0mm}
\setlength{\parsep}{\the\parskip}
\setlength{\listparindent}{\the\parindent}
%% verbatim
\setlength{\fboxsep}{5pt plus 0pt minus 0pt}



\begin{document}

\begin{frontmatter}  %

\title{A Markov-switching Model of Emerging Economies Exchange rate Volatility}

% Set to FALSE if wanting to remove title (for submission)




\author[Add1]{Nico Katzke}
\ead{nfkatzke@gmail.com}

\author[Add1,Add2]{Robert Grant}
\ead{robbiegrant1995@gmail.com}




\address[Add1]{Prescient Securities, Cape Town, South Africa}
\address[Add2]{Stellenbosch University, Stellenbosch, South Africa}

\cortext[cor]{Corresponding author: Nico Katzke}

\begin{abstract}
\small{
This paper employs a Markov-switching GARCH model to identify differing
regimes under which emerging market curreny evolves. Then, this paper
analyses how the correlation between currency volatility changes in
these differing regimes.
}
\end{abstract}

\vspace{1cm}

\begin{keyword}
\footnotesize{
Multivariate GARCH \sep Kalman Filter \sep Copula \\ \vspace{0.3cm}
\textit{JEL classification} L250 \sep L100
}
\end{keyword}
\vspace{0.5cm}
\end{frontmatter}



%________________________
% Header and Footers
%%%%%%%%%%%%%%%%%%%%%%%%%%%%%%%%%
\pagestyle{fancy}
\chead{}
\rhead{}
\lfoot{}
\rfoot{\footnotesize Page \thepage\textbackslash{}}
\lhead{}
%\rfoot{\footnotesize Page \thepage\ } % "e.g. Page 2"
\cfoot{}

%\setlength\headheight{30pt}
%%%%%%%%%%%%%%%%%%%%%%%%%%%%%%%%%
%________________________

\headsep 35pt % So that header does not go over title




\hypertarget{introduction}{%
\section{\texorpdfstring{Introduction
\label{Introduction}}{Introduction }}\label{introduction}}

The comovement of financial variables between countries is a well
researched topic. Particularly, the advent of the Global Financial
Crisis (GFC) challenged what we know about the dynamics between
international variables, with many established correlations breaking
down after the GFC. Similarly, exchange rate comovements between
emerging markets change during periods of economic uncertainty, such as
oil price shocks, economic crashes, bank and currency crises and even
across general business cycle up- and downswings.

This paper employs a Markov-switching GARCH (MS-GARCH) model on emerging
market (EM) exchange rates in order to analyse the correlation between
these currencies during periods (regimes) of high and low volatility.
MS-GARCH models, since @Hamilton(1990) introduced the concept, have
become a popular method for analysing the dynamics of financial
variables through time.

\begin{quote}
I suggest renaming the top line after @article, as done in the template
ref.bib file, to something more intuitive for you to remember. Do not
change the rest of the code. Also, be mindful of the fact that bib
references from google scholar may at times be incorrect. Reference
Latex forums for correct bibtex notation.
\end{quote}

To reference a section, you have to set a label using
``\textbackslash label'' in R, and then reference it in-text as e.g.:
section \ref{Data}.

Writing in Rmarkdown is surprizingly easy - see
\href{https://www.rstudio.com/wp-content/uploads/2015/03/rmarkdown-reference.pdf}{this
website} cheatsheet for a summary on writing Rmd writing tips.

\hypertarget{data}{%
\section{\texorpdfstring{Data \label{Data}}{Data }}\label{data}}

Exchange rate series for 41 different countries are used.\footnote{See
  appendix for full list of exchange rates} The exchange rate are
analysed at a weekly frequencty, as daily closing exchange rates create
difficult analysis due to differing time zones. All exchange rates are
shown relative to the US dollar. the period under analysis covers
1990-2019.

\begin{verbatim}
## 
## 
##       Mean   Volatiliy
## ----------  ----------
##  0.0787894   0.0113821
\end{verbatim}

\begin{verbatim}
## Specification type: Markov-switching
## Specification name: gjrGARCH_std gjrGARCH_std
## Number of parameters in each variance model: 4 4
## Number of parameters in each distribution: 1 1
## ------------------------------------------
## Fixed parameters:
## None
## ------------------------------------------
## Across regime constrained parameters:
## nu 
## ------------------------------------------
## Fitted parameters:
##          Estimate Std. Error  t value  Pr(>|t|)
## alpha0_1   0.0000     0.0000   2.9465 1.607e-03
## alpha1_1   0.0219     0.0140   1.5693 5.829e-02
## alpha2_1   0.0001     0.0002   0.4332 3.324e-01
## beta_1     0.9678     0.0138  70.2472    <1e-16
## nu_1       5.8068     0.4507  12.8848    <1e-16
## alpha0_2   0.0000     0.0000   3.5632 1.832e-04
## alpha1_2   0.2039     0.0402   5.0679 2.011e-07
## alpha2_2   0.0002     0.0007   0.2304 4.089e-01
## beta_2     0.7314     0.0461  15.8813    <1e-16
## P_1_1      0.9881     0.0050 197.4845    <1e-16
## P_2_1      0.0109     0.0063   1.7325 4.159e-02
## ------------------------------------------
## Transition matrix:
##       t+1|k=1 t+1|k=2
## t|k=1  0.9881  0.0119
## t|k=2  0.0109  0.9891
## ------------------------------------------
## Stable probabilities:
## State 1 State 2 
##  0.4783  0.5217 
## ------------------------------------------
## LL: 17939.5478
## AIC: -35857.0955
## BIC: -35784.9877
## ------------------------------------------
\end{verbatim}

\includegraphics{Template_files/figure-latex/unnamed-chunk-6-1.pdf}
\includegraphics{Template_files/figure-latex/unnamed-chunk-7-1.pdf}

\begin{figure}[H]

{\centering \includegraphics{Template_files/figure-latex/Figure1.1-1} 

}

\caption{Caption Here \label{Figure1}}\label{fig:Figure1.1}
\end{figure}

To reference the plot above, add a ``\textbackslash label'' after the
caption in the chunk heading, as done above. Then reference the plot as
such: As can be seen, figure \ref{Figure1} is excellent. The nice thing
now is that it correctly numbers all your figures (and sections or
tables) and will update if it moves. The links are also dynamic.

I very strongly suggest using ggplot2 (ideally in combination with
dplyr) using the ggtheme package to change the themes of your figures.

Also note the information that I have placed above the chunks in the
code chunks for the figures. You can edit any of these easily - visit
the Rmarkdown webpage for more information.

Here follows another figure from built-in ggplot2 data:

\begin{figure}[H]

{\centering \includegraphics{Template_files/figure-latex/figure2-1} 

}

\caption{Diamond Cut Plot \label{lit}}\label{fig:figure2}
\end{figure}

\hypertarget{methodology}{%
\section{Methodology}\label{methodology}}

This section presents a layout of a simple MS-GARCH framework, as
described in Hamilton (1990). This Markov-switching framework assumes
two distinct states under which exchange rates evolve: a high-volatility
and low-volatility. The high volatility state is denoted by
\begin{align}
S_t &=i
\end{align}

where \textbackslash begin i \&= 0,1 \textbackslash end depending on the
state. 0 denotes the low volatility regime, whereas 1 denotes the high
volatility regime \#\# Math section

Equations should be written as such:

\begin{align} 
\beta = \sum_{i = 1}^{\infty}\frac{\alpha^2}{\sigma_{t-1}^2} \label{eq1} \\ 
\int_{x = 1}^{\infty}x_{i} = 1 \notag
\end{align}

If you would like to see the equations as you type in Rmarkdown, use \$
symbols instead (see this for yourself by adjusted the equation):

\[
\beta = \sum_{i = 1}^{\infty}\frac{\alpha^2}{\sigma_{t-1}^2} \\ 
\int_{x = 1}^{\infty}x_{i} = 1
\]

Note again the reference to equation \ref{eq1}. Writing nice math
requires practice. Note I used a forward slashes to make a space in the
equations. I can also align equations using \textbf{\&}, and set to
numbering only the first line. Now I will have to type ``begin
equation'' which is a native \LaTeX command. Here follows a more
complicated equation:

\begin{align} 
    y_t &= c + B(L) y_{t-1} + e_t   \label{eq2}    \\ \notag 
    e_t &= H_t^{1/2}  z_t ; \quad z_t \sim  N(0,I_N) \quad \& \quad H_t = D_tR_tD_t \\ \notag
        D_t^2 &= {\sigma_{1,t}, \dots, \sigma_{N,t}}   \\ \notag
        \sigma_{i,t}^2 &= \gamma_i+\kappa_{i,t}  v_{i, t-1}^2 +\eta_i  \sigma_{i, t-1}^2, \quad \forall i \\ \notag
        R_{t, i, j} &= {diag(Q_{t, i, j}}^{-1}) . Q_{t, i, j} . diag(Q_{t, i, j}^{-1})  \\ \notag
        Q_{t, i, j} &= (1-\alpha-\beta)  \bar{Q} + \alpha  z_t  z_t'  + \beta  Q_{t, i, j} \notag
\end{align}

Note that in \ref{eq2} I have aligned the equations by the equal signs.
I also want only one tag, and I create spaces using ``quads''.

See if you can figure out how to do complex math using the two examples
provided in \ref{eq1} and \ref{eq2}.

\hypertarget{results}{%
\section{Results}\label{results}}

Tables can be included as follows. Use the \emph{xtable} (or kable)
package for tables. Table placement = H implies Latex tries to place the
table Here, and not on a new page (there are, however, very many ways to
skin this cat. Luckily there are many forums online!).

\begin{table}[H]
\centering
\begin{tabular}{rrrrrrrrrrrr}
  \hline
 & mpg & cyl & disp & hp & drat & wt & qsec & vs & am & gear & carb \\ 
  \hline
1 & 21.00 & 6.00 & 160.00 & 110.00 & 3.90 & 2.62 & 16.46 & 0.00 & 1.00 & 4.00 & 4.00 \\ 
  2 & 21.00 & 6.00 & 160.00 & 110.00 & 3.90 & 2.88 & 17.02 & 0.00 & 1.00 & 4.00 & 4.00 \\ 
  3 & 22.80 & 4.00 & 108.00 & 93.00 & 3.85 & 2.32 & 18.61 & 1.00 & 1.00 & 4.00 & 1.00 \\ 
  4 & 21.40 & 6.00 & 258.00 & 110.00 & 3.08 & 3.21 & 19.44 & 1.00 & 0.00 & 3.00 & 1.00 \\ 
  5 & 18.70 & 8.00 & 360.00 & 175.00 & 3.15 & 3.44 & 17.02 & 0.00 & 0.00 & 3.00 & 2.00 \\ 
   \hline
\end{tabular}
\caption{Short Table Example \label{tab1}} 
\end{table}

To reference calculations \textbf{in text}, \emph{do this:} From table
\ref{tab1} we see the average value of mpg is 20.98.

Including tables that span across pages, use the following (note that I
add below the table: ``continue on the next page''). This is a neat way
of splitting your table across a page.

Use the following default settings to build your own possibly long
tables. Note that the following will fit on one page if it can, but
cleanly spreads over multiple pages:

\begingroup\fontsize{12pt}{13pt}\selectfont
\begin{longtable}{rrrrrrrrrrr}
\caption{Long Table Example} \\ 
  \toprule
mpg & cyl & disp & hp & drat & wt & qsec & vs & am & gear & carb \\ 
  \hline 
\endhead 
\hline 
{\footnotesize Continued on next page} 
\endfoot 
\endlastfoot 
 \midrule
21.00 & 6.00 & 160.00 & 110.00 & 3.90 & 2.62 & 16.46 & 0.00 & 1.00 & 4.00 & 4.00 \\ 
  21.00 & 6.00 & 160.00 & 110.00 & 3.90 & 2.88 & 17.02 & 0.00 & 1.00 & 4.00 & 4.00 \\ 
  22.80 & 4.00 & 108.00 & 93.00 & 3.85 & 2.32 & 18.61 & 1.00 & 1.00 & 4.00 & 1.00 \\ 
  21.40 & 6.00 & 258.00 & 110.00 & 3.08 & 3.21 & 19.44 & 1.00 & 0.00 & 3.00 & 1.00 \\ 
  18.70 & 8.00 & 360.00 & 175.00 & 3.15 & 3.44 & 17.02 & 0.00 & 0.00 & 3.00 & 2.00 \\ 
  18.10 & 6.00 & 225.00 & 105.00 & 2.76 & 3.46 & 20.22 & 1.00 & 0.00 & 3.00 & 1.00 \\ 
  14.30 & 8.00 & 360.00 & 245.00 & 3.21 & 3.57 & 15.84 & 0.00 & 0.00 & 3.00 & 4.00 \\ 
  24.40 & 4.00 & 146.70 & 62.00 & 3.69 & 3.19 & 20.00 & 1.00 & 0.00 & 4.00 & 2.00 \\ 
  22.80 & 4.00 & 140.80 & 95.00 & 3.92 & 3.15 & 22.90 & 1.00 & 0.00 & 4.00 & 2.00 \\ 
  19.20 & 6.00 & 167.60 & 123.00 & 3.92 & 3.44 & 18.30 & 1.00 & 0.00 & 4.00 & 4.00 \\ 
  17.80 & 6.00 & 167.60 & 123.00 & 3.92 & 3.44 & 18.90 & 1.00 & 0.00 & 4.00 & 4.00 \\ 
  16.40 & 8.00 & 275.80 & 180.00 & 3.07 & 4.07 & 17.40 & 0.00 & 0.00 & 3.00 & 3.00 \\ 
  17.30 & 8.00 & 275.80 & 180.00 & 3.07 & 3.73 & 17.60 & 0.00 & 0.00 & 3.00 & 3.00 \\ 
  15.20 & 8.00 & 275.80 & 180.00 & 3.07 & 3.78 & 18.00 & 0.00 & 0.00 & 3.00 & 3.00 \\ 
  10.40 & 8.00 & 472.00 & 205.00 & 2.93 & 5.25 & 17.98 & 0.00 & 0.00 & 3.00 & 4.00 \\ 
  10.40 & 8.00 & 460.00 & 215.00 & 3.00 & 5.42 & 17.82 & 0.00 & 0.00 & 3.00 & 4.00 \\ 
  14.70 & 8.00 & 440.00 & 230.00 & 3.23 & 5.34 & 17.42 & 0.00 & 0.00 & 3.00 & 4.00 \\ 
  32.40 & 4.00 & 78.70 & 66.00 & 4.08 & 2.20 & 19.47 & 1.00 & 1.00 & 4.00 & 1.00 \\ 
  30.40 & 4.00 & 75.70 & 52.00 & 4.93 & 1.61 & 18.52 & 1.00 & 1.00 & 4.00 & 2.00 \\ 
  33.90 & 4.00 & 71.10 & 65.00 & 4.22 & 1.83 & 19.90 & 1.00 & 1.00 & 4.00 & 1.00 \\ 
  21.50 & 4.00 & 120.10 & 97.00 & 3.70 & 2.46 & 20.01 & 1.00 & 0.00 & 3.00 & 1.00 \\ 
  15.50 & 8.00 & 318.00 & 150.00 & 2.76 & 3.52 & 16.87 & 0.00 & 0.00 & 3.00 & 2.00 \\ 
  15.20 & 8.00 & 304.00 & 150.00 & 3.15 & 3.44 & 17.30 & 0.00 & 0.00 & 3.00 & 2.00 \\ 
  13.30 & 8.00 & 350.00 & 245.00 & 3.73 & 3.84 & 15.41 & 0.00 & 0.00 & 3.00 & 4.00 \\ 
  19.20 & 8.00 & 400.00 & 175.00 & 3.08 & 3.85 & 17.05 & 0.00 & 0.00 & 3.00 & 2.00 \\ 
  27.30 & 4.00 & 79.00 & 66.00 & 4.08 & 1.94 & 18.90 & 1.00 & 1.00 & 4.00 & 1.00 \\ 
  26.00 & 4.00 & 120.30 & 91.00 & 4.43 & 2.14 & 16.70 & 0.00 & 1.00 & 5.00 & 2.00 \\ 
  30.40 & 4.00 & 95.10 & 113.00 & 3.77 & 1.51 & 16.90 & 1.00 & 1.00 & 5.00 & 2.00 \\ 
  15.80 & 8.00 & 351.00 & 264.00 & 4.22 & 3.17 & 14.50 & 0.00 & 1.00 & 5.00 & 4.00 \\ 
  19.70 & 6.00 & 145.00 & 175.00 & 3.62 & 2.77 & 15.50 & 0.00 & 1.00 & 5.00 & 6.00 \\ 
  15.00 & 8.00 & 301.00 & 335.00 & 3.54 & 3.57 & 14.60 & 0.00 & 1.00 & 5.00 & 8.00 \\ 
  21.40 & 4.00 & 121.00 & 109.00 & 4.11 & 2.78 & 18.60 & 1.00 & 1.00 & 4.00 & 2.00 \\ 
   \bottomrule
\end{longtable}
\endgroup

\hfill

\hypertarget{huxtable}{%
\subsection{Huxtable}\label{huxtable}}

Huxtable is a very nice package for making working with tables between
Rmarkdown and Tex easier.

This cost some adjustment to the Tex templates to make it work, but it
now works nicely.

See documentation for this package
\href{https://hughjonesd.github.io/huxtable/huxtable.html}{here}. A
particularly nice addition of this package is for making the printing of
regression results a joy (see
\href{https://hughjonesd.github.io/huxtable/huxtable.html\#creating-a-regression-table}{here}).
Here follows an example:

If you are eager to use huxtable, comment out the Huxtable table in the
Rmd template, and uncomment the colortbl package in your Rmd's root.

Note that I do not include this in the ordinary template, as some latex
users have complained it breaks when they build their Rmds (especially
those using tidytex - I don't have this problem as I have the full
Miktex installed on mine). Up to you, but I strongly recommend
installing the package manually and using huxtable. To make this work,
uncomment the \emph{Adding additional latex packages} part in yaml at
the top of the Rmd file. Then comment out the huxtable example in the
template below this line. Reknit, and enjoy.

FYI - R also recently introduced the gt package, which is worthwhile
exploring too.

\hypertarget{lists}{%
\section{Lists}\label{lists}}

To add lists, simply using the following notation

\begin{itemize}
\item
  This is really simple

  \begin{itemize}
  \tightlist
  \item
    Just note the spaces here - writing in R you have to sometimes be
    pedantic about spaces\ldots{}
  \end{itemize}
\item
  Note that Rmarkdown notation removes the pain of defining
  \LaTeX environments!
\end{itemize}

\hypertarget{conclusion}{%
\section{Conclusion}\label{conclusion}}

I hope you find this template useful. Remember, stackoverflow is your
friend - use it to find answers to questions. Feel free to write me a
mail if you have any questions regarding the use of this package. To
cite this package, simply type citation(``Texevier'') in Rstudio to get
the citation for Katzke (\protect\hyperlink{ref-Texevier}{2017}) (Note
that uncited references in your bibtex file will not be included in
References).

\hypertarget{references}{%
\section*{References}\label{references}}
\addcontentsline{toc}{section}{References}

\hypertarget{refs}{}
\leavevmode\hypertarget{ref-Texevier}{}%
Katzke, N. F. 2017. \emph{Texevier: Package to Create Elsevier Templates
for Rmarkdown}. Stellenbosch, South Africa: Bureau for Economic
Research.

% Force include bibliography in my chosen format:

\bibliographystyle{Tex/Texevier}
\bibliography{Tex/ref}





\end{document}
